\section{Basis}

Kugelkoordinaten \cite[Glg. 2.28]{qm}
\begin{gather*}
	\vec{n} = \begin{pmatrix} \sin\theta\cos\varphi & \sin\theta\sin\varphi & \cos\theta \end{pmatrix}^T \\
	\ket{-\vec{n}}=\sin\frac{\theta}{2}\ket{+z}-e^{i\varphi}\cos\frac{\theta}{2}\ket{-z} \qquad \ket{+\vec{n}}=\cos\frac{\theta}{2}\ket{+z}+e^{i\varphi}\sin \frac{\theta}{2}\ket{-z}
\end{gather*}
Orthonormalität \& Vollständigkeit \cite[Glg. 2.21, 2.23]{qm}
\begin{equation*}
    \braket{\pm\vec{n}|\pm\vec{n}}=1 \quad \braket{\pm\vec{n}|\mp\vec{n}}=0 \qquad \ketbra{-\vec{n}}{-\vec{n}}+\ketbra{+\vec{n}}{+\vec{n}}=1
\end{equation*}
Skalarprodukte \cite[Folg. aus Tab. nach Glg. 2.28,]{qm}
\begin{equation*}
    \begin{array}{c | ccc ccc}
                 & \ket{-x}            & \ket{-y}            & \ket{-z}            & \ket{+x}            & \ket{+y}            & \ket{+z} \\
        \hline
        \bra{-x} & 1                   &\frac{1+i}{2}        & \frac{-1}{\sqrt{2}} & 0                   & \frac{i-i}{2}       & \frac{+1}{\sqrt{2}} \\[3pt]
        \bra{-y} & \frac{1-i}{2}       & 1                   & \frac{+i}{\sqrt{2}} & \frac{1+i}{2}       & 0                   & \frac{+1}{\sqrt{2}} \\[3pt]
        \bra{-z} & \frac{-1}{\sqrt{2}} & \frac{-i}{\sqrt{2}} & 1                   & \frac{+1}{\sqrt{2}} & \frac{+i}{\sqrt{2}} & 0 \\[3pt]
        \bra{+x} & 0                   & \frac{1-i}{2}       & \frac{+1}{\sqrt{2}} & 1                   & \frac{1+i}{2}       & \frac{+1}{\sqrt{2}} \\[3pt]
        \bra{+y} & \frac{1+i}{2}       & 0                   & \frac{-i}{\sqrt{2}} & \frac{1-i}{2}       & 1                   & \frac{+1}{\sqrt{2}} \\[3pt]
        \bra{+z} & \frac{+1}{\sqrt{2}} & \frac{+1}{\sqrt{2}} & 0                   & \frac{+1}{\sqrt{2}} & \frac{+1}{\sqrt{2}} & 1 \\
    \end{array}
\end{equation*}
Darstellungen in allen Basen \cite[Tab. nach Glg. 2.28 mit Folgerungen]{qm}
\begin{equation*}
    \begin{array}{c | c c | c @{\ =\ } c @{\ =\ } c}
        \vec{n} & \theta        & \varphi       & \multicolumn{3}{c}{\ket{\pm\vec{n}}} \\
        \hline
        x       & \frac{\pi}{2} & 0             & \ket{\pm x}                                        & \frac{1\mp i}{2}\ket{+y}+\frac{1\pm i}{2}\ket{-y}     & \frac{1}{\sqrt{2}}\left(\ket{+z}\pm\ket{-z}\right) \\[6pt]
        y       & \frac{\pi}{2} & \frac{\pi}{2} & \frac{1\pm i}{2}\ket{+x}+\frac{1\mp i}{2}\ket{-x}  & \ket{\pm y}                                           & \frac{1}{\sqrt{2}}\left(\ket{+z}\pm i\ket{-z}\right) \\[6pt]
        z       & 0             & \pi           & \frac{1}{\sqrt{2}}\left(\ket{+x}\pm\ket{-x}\right) & \frac{1}{1|i\sqrt{2}}\left(\ket{+y}\pm\ket{-y}\right) & \ket{\pm z}
    \end{array}
\end{equation*}

\section{Operatoren}

Spektraldarstellung \cite[Glg. 2.30]{qm}
\begin{equation*}
	\hat{S}_{\vec{n}} = \frac{\hbar}{2}\left(\ketbra{+\vec{n}}{+\vec{n}}-\ketbra{-\vec{n}}{-\vec{n}}\right)
\end{equation*}
Eigenschaften \cite[Glg. 2.35, 2.34]{qm}
\begin{equation*}
	\hat{S}_{\vec{n}}^2 = \frac{\hbar^2}{4} \ \text{(involut proportional)} \qquad \left[\hat{S}_i,\hat{S}_j\right] = i\hbar\epsilon_{ijk}\hat{S}_\gamma
\end{equation*}
Matrixdarstellung in $z$-Basis \cite[Glg. 2.31]{qm}
\begin{equation*}
    \hat{S}_x^{(z)} = \frac{\hbar}{2} \sigma_x \qquad \hat{S}_y^{(z)} = \frac{\hbar}{2} \sigma_y \qquad \hat{S}_z^{(z)} = \frac{\hbar}{2} \sigma_z
\end{equation*}

\section{Pauli-Matrizen}

\begin{itemize}
    \item Indizes: $i,j,k\in\{1, 2, 3\}, \ \alpha,\beta,\gamma\in\{0,1,2,3\}$ \begin{refsection}\footfullcite{fliessbach}\end{refsection}
\end{itemize}
Definition \cite[Glg. 2.31, S.52]{qm}
\begin{equation*}
	\sigma_0=\begin{pmatrix} 1 & 0 \\ 0 & 1 \end{pmatrix} \qquad \sigma_x=\begin{pmatrix} 0 & 1 \\ 1 & 0 \end{pmatrix} \qquad \sigma_y=\begin{pmatrix} 0 & -i \\ i & 0 \end{pmatrix} \qquad \sigma_z=\begin{pmatrix}1 & 0 \\ 0 & -1 \end{pmatrix}
\end{equation*}
Basis aller 2x2-Matrizen \cite[Übung]{qm}
\begin{equation*}
	\begin{pmatrix} a_{00} & a_{01} \\ a_{10} & a_{11} \end{pmatrix} = \frac{a_{00}+a_{11}}{2}\sigma_0 + \frac{a_{01}+a_{10}}{2}\sigma_x + i\frac{a_{01}-a_{10}}{2}\sigma_y + \frac{a_{00}-a_{11}}{2}\sigma_z
\end{equation*}
Eigenschaften \cite[Glg. 2.32]{qm}
\begin{align*}
        \sigma_\alpha^\dagger &= \sigma_\alpha \ \text{(hermitesch)}  & \det\sigma_i       &= -1         & \sigma_i\sigma_j &= \delta_{ij} + i\epsilon_{ijk}\sigma_k \\[6pt]
        \sigma_\alpha^\dagger &= \sigma_\alpha^{-1} \ \text{(unitär)} & \tr\sigma_i        &= 0          & [\sigma_i, \sigma_j] &= 2i\epsilon_{ijk}\sigma_k \\[6pt]
        \sigma_\alpha^2 &= 1 \ \text{(involut)}                       & \sigma_p(\sigma_i) &= \{-1, +1\} & \{\sigma_i, \sigma_j\} &= 2\delta_{ij}
\end{align*}
