\documentclass[12pt,a4paper]{report}

\usepackage{labor}



\title{Quantenmechanik Formelsammlung}
\author{Gössl, Sebastian}
\date{19WS-22SS}



\begin{document}



\maketitle



\begin{abstract}
    Dies ist eine grobe Zusammenfassung der Formeln aus der Vorlesung und zugehörigen Übung Quantenmechanik, veranstaltet von Evertz, Hans \& Gazzaneo, Paolo. Hauptliteratur ist das lehrreiche Vorlesungsskript \cite{qm}, auf welches immer für tiefere Einblicke und Kontext als Quelle verwiesen wird. \\
    Ziel dieses Dokumentes ist es ein Arbeitsmittel des Übungsalltages und kein Lernmittel zu sein. Die Formeln sind so abstrakt gehalten, dass sie für möglichst viele Fälle direkt anwendbar sind; unkonkrete Konzepte \& irrelevante Spezialfälle werden nicht übernommen. \\
    Bei Fragen, Anregungen oder Verbesserungsvorschlägen, egal wie insignifikant diese scheinen mögen, bitte melden.
\end{abstract}



\tableofcontents



\nocite{nolting51}
\nocite{nolting52}

\printbibliography[keyword=major,heading=bibintoc]
\printbibliography[keyword=further,heading=subbibliography,title={Weiterführend}]



\chapter{Zustände}

Wahrscheinlichkeit von Messergebnis \cite[Glg. 2.12]{qm}
\begin{equation*}
	W(a_j\mid\psi)=|\braket{a_j|\psi}|^2 \qquad \text{wobei} \ A=\sum_ja_j\ketbra{a_j}{a_j}
\end{equation*}
Erwartungswert \cite[Glg. 2.16, S. 33]{qm}
\begin{equation*}
	\braket{\hat{A}}_\psi = \braket{\psi|\hat{A}|\psi} = \sum_j|\braket{a_j|\psi}|^2a_j
\end{equation*}
Gebundenheit \cite[Kap. 4.1.1]{qm}
\begin{itemize}
    \item Gebundene Zustände $\Leftrightarrow$ normierbare Eigenzustände von $\hat{H}$
    \begin{itemize}
        \item Diskretes Energiespektrum
    \end{itemize}
    \item Unebundene Zustände $\Leftrightarrow$ nicht normierbare Eigenzustände von $\hat{H}$
    \begin{itemize}
        \item Kontinuierliches Energiespektrum
    \end{itemize}
\end{itemize}
Untere Energieschranke \cite[Glg. 4.31]{qm}
\begin{equation*}
	E > \min_{\vec{x}}V(\vec{x})
\end{equation*}

\section{Teilchenströme}

Wahrscheinlichkeitsstromdichte \& Kontinuität \cite[Glg. 4.38]{qm}
\begin{gather*}
	\vec{j} = \frac{\hbar}{m}\text{Im}\ \psi^*(\vec{x},t)\vec{\nabla}\psi(\vec{x},t) \\
	\frac{\partial}{\partial t}\rho(\vec{x},t) = -\vec{\nabla}\vec{j}(\vec{x},t)
\end{gather*}
Transmissions- \& Reflexionskoeffizient \cite[S. 116]{qm}
\begin{equation*}
	T + R = 1
\end{equation*}

\section{Dichteoperator}

Statistischer Operator \cite[Glg. 9.5, 9.9, 9.8]{qm}
\begin{equation*}
	\hat{\rho} = \sum_j\rho_j\ketbra{\varphi_j}{\varphi_j} \qquad \tr\hat{\rho}=1 \qquad \braket{\hat{A}}_{\hat{\rho}}=\tr\hat{\rho}\hat{A}
\end{equation*}
Hinreichende Charakterisierung eines reinen Zustandes \cite[Glg. 9.10, 9.12, 9.13]{qm}
\begin{equation*}
	\hat{\rho}=\ketbra{\psi}{\psi} \qquad \hat{\rho}^2=\hat{\rho} \ \text{(Idempotenz)} \qquad \tr\hat{\rho}=1
\end{equation*}
Schrödingergleichung \cite[Glg. 9.17]{qm}
\begin{equation*}
	i\hbar\frac{d}{dt}\hat{\rho}(t) = \left[\hat{H},\hat{\rho}(t)\right]
\end{equation*}
Reduzierte Dichtematrix \cite[Glg. 9.23]{qm}
\begin{equation*}
	\tilde{\rho}_{ij} = \sum_k\rho_{kkij}
\end{equation*}


\chapter{Operatoren}

Korrespondenzprinzip \cite[Glg. 2.13]{qm}
\begin{itemize}
    \item Observablen sind hermitsche Operatoren
\end{itemize}
Dualraum \cite[Glg. A.32]{qm}
\begin{equation*}
	\hat{A}\ket{v}=\ket{w} \qquad \Leftrightarrow \qquad \bra{v}\hat{A}^\dagger=\bra{w}
\end{equation*}
Matrixdarstellung \cite[S. A.16]{qm}
\begin{equation*}
	\hat{O} = \sum_{ij}O_{ij}^{(e)}\ketbra{e_i}{e_j} \qquad \Leftrightarrow \qquad O_{ij}^{(e)}=\braket{e_i|\hat{O}|e_j}
\end{equation*}
Spur \cite[Def. A.19, Thm. A.4]{qm}
\begin{equation*}
	\tr\hat{O} = \sum_i\braket{e_i|\hat{O}|e_i} \qquad \text{(basisunabhängig)}
\end{equation*}
Kommutator, Antikommutator \cite[Glg. A.22, S. 51]{qm}
\begin{equation*}
	[\hat{A},\hat{B}]=\hat{A}\hat{B}-\hat{B}\hat{A} \qquad \{\hat{A},\hat{B}\}=\hat{A}\hat{B}+\hat{B}\hat{A}
\end{equation*}
Produktregel \cite[Glg. A.23]{qm}
\begin{equation*}
	[\hat{A}\hat{B},\hat{C}] = \hat{A}[\hat{B},\hat{C}] + [\hat{A},\hat{C}]\hat{B}
\end{equation*}
Unbestimmtheitsrelation \cite[Glg. A.160]{qm}
\begin{equation*}
	\braket{(\Delta\hat{A})^2}\braket{(\Delta\hat{B})^2} \geq \frac{|\braket{[\hat{A},\hat{B}]}|^2}{4}
\end{equation*}


\chapter{Zeitenwicklung \& Hamiltonoperator}

\section{Zeitentwicklung}

Definition \cite[Glg. 3.2]{qm}
\begin{equation*}
    \ket{\psi(t)} = \hat{U}(t, t_0)\ket{\psi(t_0)}
\end{equation*}
Eigenschaften \cite[S. 56, Glg 3.3, 3.4]{qm}
\begin{align*}
	\hat{U}(t_0, t_0) &= 1 \ (\text{Kontinuität}) \\
	\hat{U}(t, t_0)^\dagger &= \hat{U}(t, t_0)^{-1}=\hat{U}(t_0, t) \ (\text{Unitarität}) \\
	\hat{U}(t_2, t_0) &= \hat{U}(t_2, t_1)\hat{U}(t_1, t_0) \ (\text{''Propagatoreigeschaft''})
\end{align*}
Taylorentwicklung \cite[Glg. 3.4]{qm}
\begin{equation*}
    \hat{U}(t+dt,t) = 1-\frac{i}{\hbar}\hat{H}(t)dt+\cdots
\end{equation*}
Schrödingergleichung für den Zeitentwicklungsoperator und den Zustand \cite[Glg. 3.5, 3.15]{qm}
\begin{equation*}
    i\hbar\frac{d}{dt}\hat{U}(t,t_0)=\hat{H}(t)\hat{U}(t,t_0) \qquad \Leftrightarrow \qquad  i\hbar\frac{d}{dt}\ket{\psi(t)}=\hat{H}\ket{\psi(t)}
\end{equation*}
Formale Lösung für kommutierende Hamiltonians ($[\hat{H}(t_1), \hat{H}(t_2)]=0$), Spektraldarstellung \cite[Glg. 3.9]{qm}
\begin{equation*}
    \hat{U}(t, t_0) = e^{-\frac{i}{\hbar}\int_{t_0}^t\hat{H}(\tau)d\tau} = \sum_n e^{-\frac{i}{\hbar}\int_{t_0}^tE_n(\tau)d\tau}\ketbra{\varphi_n}{\varphi_n}
\end{equation*}
Dysonscher Zeitordnungsoperator \cite[Glg. A.196]{qm}
\begin{equation*}
	\mathcal{T}\left(\hat{H}(t_1),\hat{H}(t_2)\right) = \begin{cases}
		\hat{H}(t_1)\hat{H}(t_2) & t_1 \geq t_2 \\
		\hat{H}(t_2)\hat{H}(t_1) & t_2 \geq t_1
	\end{cases}
\end{equation*}
Von Neumannsche Reihe für den Zeitentwicklungsoperator bei nicht notwendigerweise kommutierenden Hamiltonoperatoren \cite[Glg. A.194]{qm}
\begin{align*}
	\hat{U}(t,t_0) &= \sum_{n=0}^\infty\left(\frac{-i}{\hbar}\right)^n\int_{t_0}^t\int_{t_0}^{t_1}\cdots\int_{t_0}^{t_{n-1}}\hat{H}(t_1)\cdots\hat{H}(t_n)dt_n\cdots dt_1 \\
	&= \sum_{n=0}^\infty\frac{1}{n!}\left(\frac{-i}{\hbar}\right)^n\int_{t_0}^t\int_{t_0}^{t_1}\cdots\int_{t_0}^{t_{n-1}}\mathcal{T}\left(\hat{H}(t_1),\dots,\hat{H}(t_n)\right)dt_n\cdots dt_1 \\
	&= \mathcal{T}e^{-\frac{i}{\hbar}\int_{t_0}^t\hat{H}(\tau)d\tau}
\end{align*}

\section{Hamiltonoperator}

Hamiltonian = Energieobservable, Spektraldarstellung für kommutierende Hamiltonians ($[\hat{H}(t_1), \hat{H}(t_2)]=0$) \cite[Glg. 3.6]{qm}
\begin{equation*}
	E=\braket{\psi|\hat{H}|\psi} \qquad \hat{H}(t)=\sum_n E_n(t)\ketbra{\varphi_n}{\varphi_n}
\end{equation*}
Stationäre Schrödingergleichung \cite[3.36]{qm}
\begin{equation*}
    \hat{H}\ket{\varphi_n} = E_n\ket{\varphi_n}
\end{equation*}
Korrespondenzprinzip, wichtige Hamiltonians \cite[S. 59, Glg. 3.10-12]{qm}
\begin{align*}
	\hat{H} &= H(x\to\hat{Q},p\to\hat{P}) \\
	\hat{H}(t) &= \frac{\hat{\vec{P}}^2}{2m}+\hat{V}(\hat{\vec{Q}},t) && \text{Teilchen im äußeren Potential} \\
	\hat{H}(t) &= \frac{(\hat{P}-e\hat{\vec{A}})^2}{2m}+e\varphi(\hat{\vec{Q}},t) && \text{Geladenes Teilchen im äußeren elektromag. Feld} \\
	\hat{H}(t) &= -\mu\hat{\vec{B}}\hat{\vec{S}} && \text{Neutrales Spin-$1/2$-Teilchen im Magnetfeld}
\end{align*}
Teilchen im äußeren Potential im Ortsraum \cite[Glg. 3.31]{qm}
\begin{equation*}
	i\hbar\frac{d}{dt}\psi(\vec{x},t) = \left(-\frac{\hbar^2}{2m}\vec{\nabla} + V(\vec{x},t)\right)\psi(\vec{x},t)
\end{equation*}
Zeitabhängigkeit von Erwartungswerten \cite[Glg. 3.42]{qm}
\begin{equation*}
    \frac{d}{dt}\braket{\hat{A}(t)} = \frac{i}{\hbar} \braket{[\hat{H}(t), \hat{A}(t)]} + \braket{\frac{d\hat{A}(t)}{dt}}
\end{equation*}
Hellman-Feynman-Theorem \cite[Kap. A.12.3]{qm}
\begin{equation*}
	\frac{d}{d\lambda}E_n(\lambda) = \braket{\varphi_n(\lambda) | \frac{d\hat{H}(\lambda)}{d\lambda} | \varphi_n(\lambda)}
\end{equation*}

\section{Heisenberg-Bild}

Definition \cite[Glg. 3.46]{qm}
\begin{equation*}
    \ket{\psi^H(t)} = \ket{\psi^S(t_0)} \qquad \hat{O}^H(t) = \hat{U}^\dagger(t, t_0)\hat{O}^S\hat{U}(t, t_0)
\end{equation*}
Bewegungsgleichung \cite[Glg. 3.51]{qm}
\begin{equation*}
    \frac{d}{dt}\hat{Q}^H(t) = \hat{U}^\dagger(t, t_0)\left( \frac{i}{\hbar}[\hat{H}^S(t), \hat{O}^S(t)] + \frac{d}{dt}\hat{O}^S(t)\right) \hat{U}(t, t_0)
\end{equation*}
Newtonische Bewegungsgleichung für den Ortsoperator \cite[Glg. 3.55]{qm}
\begin{equation*}
    m\frac{d^2}{dt^2}\hat{\vec{X}}^H(t) = -\vec{\nabla}\hat{V}(\hat{\vec{X}}^H)
\end{equation*}
Ehrenfest-Theorem \cite[Glg. 3.56]{qm}
\begin{equation*}
    m\frac{d^2}{dt^2}\braket{\hat{\vec{X}}} = \braket{-\vec{\nabla}V(\hat{\vec{X}})}
\end{equation*}


\chapter{Spin 1/2}

\section{Basis}

Kugelkoordinaten \cite[Glg. 2.28]{qm}
\begin{gather*}
	\vec{n} = \begin{pmatrix} \sin\theta\cos\varphi & \sin\theta\sin\varphi & \cos\theta \end{pmatrix}^T \\
	\ket{-\vec{n}}=\sin\frac{\theta}{2}\ket{+z}-e^{i\varphi}\cos\frac{\theta}{2}\ket{-z} \qquad \ket{+\vec{n}}=\cos\frac{\theta}{2}\ket{+z}+e^{i\varphi}\sin \frac{\theta}{2}\ket{-z}
\end{gather*}
Orthonormalität \& Vollständigkeit \cite[Glg. 2.21, 2.23]{qm}
\begin{equation*}
    \braket{\pm\vec{n}|\pm\vec{n}}=1 \quad \braket{\pm\vec{n}|\mp\vec{n}}=0 \qquad \ketbra{-\vec{n}}{-\vec{n}}+\ketbra{+\vec{n}}{+\vec{n}}=1
\end{equation*}
Skalarprodukte \cite[Folg. aus Tab. nach Glg. 2.28,]{qm}
\begin{equation*}
    \begin{array}{c | ccc ccc}
                 & \ket{-x}            & \ket{-y}            & \ket{-z}            & \ket{+x}            & \ket{+y}            & \ket{+z} \\
        \hline
        \bra{-x} & 1                   &\frac{1+i}{2}        & \frac{-1}{\sqrt{2}} & 0                   & \frac{i-i}{2}       & \frac{+1}{\sqrt{2}} \\[3pt]
        \bra{-y} & \frac{1-i}{2}       & 1                   & \frac{+i}{\sqrt{2}} & \frac{1+i}{2}       & 0                   & \frac{+1}{\sqrt{2}} \\[3pt]
        \bra{-z} & \frac{-1}{\sqrt{2}} & \frac{-i}{\sqrt{2}} & 1                   & \frac{+1}{\sqrt{2}} & \frac{+i}{\sqrt{2}} & 0 \\[3pt]
        \bra{+x} & 0                   & \frac{1-i}{2}       & \frac{+1}{\sqrt{2}} & 1                   & \frac{1+i}{2}       & \frac{+1}{\sqrt{2}} \\[3pt]
        \bra{+y} & \frac{1+i}{2}       & 0                   & \frac{-i}{\sqrt{2}} & \frac{1-i}{2}       & 1                   & \frac{+1}{\sqrt{2}} \\[3pt]
        \bra{+z} & \frac{+1}{\sqrt{2}} & \frac{+1}{\sqrt{2}} & 0                   & \frac{+1}{\sqrt{2}} & \frac{+1}{\sqrt{2}} & 1 \\
    \end{array}
\end{equation*}
Darstellungen in allen Basen \cite[Tab. nach Glg. 2.28 mit Folgerungen]{qm}
\begin{equation*}
    \begin{array}{c | c c | c @{\ =\ } c @{\ =\ } c}
        \vec{n} & \theta        & \varphi       & \multicolumn{3}{c}{\ket{\pm\vec{n}}} \\
        \hline
        x       & \frac{\pi}{2} & 0             & \ket{\pm x}                                        & \frac{1\mp i}{2}\ket{+y}+\frac{1\pm i}{2}\ket{-y}     & \frac{1}{\sqrt{2}}\left(\ket{+z}\pm\ket{-z}\right) \\[6pt]
        y       & \frac{\pi}{2} & \frac{\pi}{2} & \frac{1\pm i}{2}\ket{+x}+\frac{1\mp i}{2}\ket{-x}  & \ket{\pm y}                                           & \frac{1}{\sqrt{2}}\left(\ket{+z}\pm i\ket{-z}\right) \\[6pt]
        z       & 0             & \pi           & \frac{1}{\sqrt{2}}\left(\ket{+x}\pm\ket{-x}\right) & \frac{1}{1|i\sqrt{2}}\left(\ket{+y}\pm\ket{-y}\right) & \ket{\pm z}
    \end{array}
\end{equation*}

\section{Operatoren}

Spektraldarstellung \cite[Glg. 2.30]{qm}
\begin{equation*}
	\hat{S}_{\vec{n}} = \frac{\hbar}{2}\left(\ketbra{+\vec{n}}{+\vec{n}}-\ketbra{-\vec{n}}{-\vec{n}}\right)
\end{equation*}
Eigenschaften \cite[Glg. 2.35, 2.34]{qm}
\begin{equation*}
	\hat{S}_{\vec{n}}^2 = \frac{\hbar^2}{4} \ \text{(involut proportional)} \qquad \left[\hat{S}_i,\hat{S}_j\right] = i\hbar\epsilon_{ijk}\hat{S}_\gamma
\end{equation*}
Matrixdarstellung in $z$-Basis \cite[Glg. 2.31]{qm}
\begin{equation*}
    \hat{S}_x^{(z)} = \frac{\hbar}{2} \sigma_x \qquad \hat{S}_y^{(z)} = \frac{\hbar}{2} \sigma_y \qquad \hat{S}_z^{(z)} = \frac{\hbar}{2} \sigma_z
\end{equation*}

\section{Pauli-Matrizen}

\begin{itemize}
    \item Indizes: $i,j,k\in\{1, 2, 3\}, \ \alpha,\beta,\gamma\in\{0,1,2,3\}$ \begin{refsection}\footfullcite{fliessbach}\end{refsection}
\end{itemize}
Definition \cite[Glg. 2.31, S.52]{qm}
\begin{equation*}
	\sigma_0=\begin{pmatrix} 1 & 0 \\ 0 & 1 \end{pmatrix} \qquad \sigma_x=\begin{pmatrix} 0 & 1 \\ 1 & 0 \end{pmatrix} \qquad \sigma_y=\begin{pmatrix} 0 & -i \\ i & 0 \end{pmatrix} \qquad \sigma_z=\begin{pmatrix}1 & 0 \\ 0 & -1 \end{pmatrix}
\end{equation*}
Basis aller 2x2-Matrizen \cite[Übung]{qm}
\begin{equation*}
	\begin{pmatrix} a_{00} & a_{01} \\ a_{10} & a_{11} \end{pmatrix} = \frac{a_{00}+a_{11}}{2}\sigma_0 + \frac{a_{01}+a_{10}}{2}\sigma_x + i\frac{a_{01}-a_{10}}{2}\sigma_y + \frac{a_{00}-a_{11}}{2}\sigma_z
\end{equation*}
Eigenschaften \cite[Glg. 2.32]{qm}
\begin{align*}
        \sigma_\alpha^\dagger &= \sigma_\alpha \ \text{(hermitesch)}  & \det\sigma_i       &= -1         & \sigma_i\sigma_j &= \delta_{ij} + i\epsilon_{ijk}\sigma_k \\[6pt]
        \sigma_\alpha^\dagger &= \sigma_\alpha^{-1} \ \text{(unitär)} & \tr\sigma_i        &= 0          & [\sigma_i, \sigma_j] &= 2i\epsilon_{ijk}\sigma_k \\[6pt]
        \sigma_\alpha^2 &= 1 \ \text{(involut)}                       & \sigma_p(\sigma_i) &= \{-1, +1\} & \{\sigma_i, \sigma_j\} &= 2\delta_{ij}
\end{align*}


\chapter{Orts- \& Impulsraum}


\begin{itemize}
    \item Im diskreten Fall sind die Integrale durch Summen und die Delta-Distributionen durch Kronecker-Deltas zu ersetzen.
    \item Für $n$ ist die Dimensionalität zu verwenden.
\end{itemize}
Spektraldarstellung \cite[Glg. A.96, A.99, A.127, A.128]{qm}
\begin{align*}
	\hat{\vec{Q}} &= \int_{\mathbb{R}^n}\vec{x}\ketbra{\vec{x}}{\vec{x}}d\vec{x} & \hat{\vec{Q}}\ket{\vec{x}} &= \vec{x}\ket{\vec{x}} \\
	\hat{\vec{P}} &= \int_{\mathbb{R}^n}\vec{p}\ketbra{\vec{p}}{\vec{p}}d\vec{p} & \hat{\vec{P}}\ket{\vec{p}} &= \vec{p}\ket{\vec{p}}
\end{align*}
Basis (Orthonormalität \& Vollständigkeit) \cite[Glg. A.85, A.86, A.122, A.124, A.106, A.108]{qm}
\begin{align*}
	\braket{\vec{x}|\vec{y}} &= \delta(\vec{x}-\vec{y}) & \int_{\mathbb{R}^n}\ketbra{\vec{x}}{\vec{x}}d\vec{x} &= 1 \\
	\braket{\vec{p}|\vec{q}} &= \delta(\vec{p}-\vec{q}) & \int_{\mathbb{R}^n}\ketbra{\vec{p}}{\vec{p}}d\vec{p} &= 1 \\
	\braket{\vec{k}|\vec{l}} &= \delta(\vec{k}-\vec{l}) & \int_{\mathbb{R}^n}\ketbra{\vec{k}}{\vec{k}}d\vec{k} &= 1
\end{align*}
Wirkung untereinander \cite[Glg. A.129, S. A.56]{qm}
\begin{equation*}
	\braket{\vec{x}|\vec{p}} = \frac{1}{\sqrt{2\pi\hbar}^n}e^{\frac{i}{\hbar}\vec{x}\vec{p}} \qquad \braket{\vec{x}|\vec{k}} = \frac{1}{\sqrt{2\pi}^n}e^{i\vec{x}\vec{k}}
\end{equation*}
Kommutator \cite[Glg. A.151, A.152, A.153]{qm}
\begin{equation*}
	\left[\hat{Q}_j,\hat{P}_k\right] = i\hbar\delta_{ij} \qquad \left[\hat{Q}_j,f(\hat{\vec{Q}},\hat{\vec{P}})\right] = i\hbar\frac{\partial}{\partial\hat{P}_j}f(\hat{\vec{Q}},\hat{\vec{P}}) \qquad \left[f(\hat{\vec{Q}},\hat{\vec{P}}),\hat{P}_j\right] = i\hbar\frac{\partial}{\partial\hat{Q}_j}f(\hat{\vec{Q}},\hat{\vec{P}})
\end{equation*}
Wirkung auf Wellenfunktion \cite[Glg. A.83, A.125, A.113]{qm}
\begin{equation*}
	\braket{\vec{x}|f} = f(\vec{x}) \qquad \braket{\vec{p}|f} = \frac{1}{\sqrt{\hbar}^n}\mathcal{F}[f](\vec{p}) \qquad \braket{\vec{k}|f} = \mathcal{F}[f](\vec{k})
\end{equation*}
Darstellung in anderen Räumen \cite[Glg. A.132, eigene Folgerung, S. A.60]{qm}
\begin{equation*}
	\hat{\vec{P}} = -i\hbar\int_{\mathbb{R}^n}\ket{\vec{x}}\vec{\nabla}\bra{\vec{x}}d\vec{x} \qquad \hat{\vec{Q}} = +i\hbar\int_{\mathbb{R}^n}\ket{\vec{p}}\vec{\nabla}\bra{\vec{p}}d\vec{p} \qquad \ket{\vec{p}} = \frac{1}{\sqrt{\hbar}^n}\ket{\vec{k}}
\end{equation*}


\chapter{Harmonischer Oszillator}

Hamiltonian \cite[Glg. 4.61, 4.65]{qm}
\begin{equation*}
	\hat{H} = \frac{1}{2m}\hat{P}^2+\frac{\omega^2 m}{2}\hat{Q}^2 = \hbar\omega\left(\hat{N}+\frac{1}{2}\right)
\end{equation*}
Leiteroperatoren (Erzeugungs- \& Vernichtungsoperator) \cite[Glg. 4.62]{qm}
\begin{align*}
	\hat{a}^\dagger &= \sqrt{\frac{m\omega}{2\hbar}}\left(\hat{Q}-i\frac{\hat{P}}{m\omega}\right) & \hat{a} &= \sqrt{\frac{m\omega}{2\hbar}}\left(\hat{Q}+i\frac{\hat{P}}{m\omega}\right) \\
	\hat{Q} &= \sqrt{\frac{\hbar}{2m\omega}}\left(\hat{a}^\dagger+\hat{a}\right) & \hat{P} &= i\sqrt{\frac{\hbar m\omega}{2}}\left(\hat{a}^\dagger-\hat{a}\right)
\end{align*}
Anzahloperator \cite[Glg. 4.64, 4.75]{qm}
\begin{equation*}
	\hat{N} = \hat{a}^\dagger\hat{a} = \sum_{n\in\mathbb{N}_0} n\ketbra{n}{n}
\end{equation*}
Wirkung \cite[Glg. 4.72, 4.73, 4.75, 4.77]{qm}
\begin{equation*}
    \hat{a}^\dagger\ket{n}=\sqrt{n+1}\ket{n+1} \qquad \hat{a}\ket{n}=\sqrt{n}\ket{n-1} \qquad \hat{N}\ket{n}=n\ket{n} \qquad \ket{n}=\frac{(\hat{a}^\dagger)^n}{\sqrt{n!}}\ket{0}
\end{equation*}
Eigenschaften \cite[Glg. 4.66, 4.67]{qm}
\begin{equation*}
    [\hat{a}, \hat{a}^\dagger] = 1 \qquad [\hat{N}, \hat{a}^\dagger] = \hat{a}^\dagger \qquad [\hat{N}, \hat{a}] = -\hat{a}
\end{equation*}
Erwartungswerte \& charakteristische Skala \cite[Glg. 4.82, nach Glg. 4.79]{qm}
\begin{align*}
	\braket{n|\hat{Q}|n} &= 0 & \braket{n|(\Delta\hat{Q})^2|n} &= \frac{x_0^2}{2}(2n+1) & x_0 &= \sqrt{\frac{\hbar}{m\omega}} \\
	\braket{n|\hat{P}|n} &= 0 & \braket{n|(\Delta\hat{P})^2|n} &= \frac{p_0^2}{2}(2n+1) & p_0 &= \sqrt{\hbar m\omega}
\end{align*}
Lösung im Ortsraum \cite[Glg. 4.86]{qm}
\begin{itemize}
    \item Hermite-Funktionen im Kapitel Wellenfunktionen
\end{itemize}
Mehrdimensionaler Harmonischer Oszillator \cite[Kap. 4.4.6]{nolting51}
\begin{equation*}
    \hat{H}=\sum_j\hat{H}_j \qquad E_{\vec{n}}=\sum_jE_{n_j} \qquad \ket{\vec{n}}=\prod_j\ket{n_j}\qquad \text{(separierbar)}
\end{equation*}

\section{Kohärente Zustände}

Definition \cite[Glg. 4.95, 4.97]{qm}
\begin{equation*}
    \hat{a}\ket{\lambda}=\lambda\ket{\lambda} \qquad \Leftrightarrow \qquad \bra{\lambda}\hat{a}^\dagger=\bra{\lambda}\lambda^*
\end{equation*}
Darstellung \cite[Glg. 4.96]{qm}
\begin{equation*}
    \ket{\lambda} = e^{\frac{-|\lambda|^2}{2}} \sum_{n\in\mathbb{N}_0}\frac{\lambda^n}{\sqrt{n!}}\ket{n}
\end{equation*}


\chapter{Symmetrien}

\section{Nöthersches Theorem}

Aussage \& Beispiele \cite[S. 192]{qm}
\begin{align*}
		\text{System invariant unter $e^{-\frac{i}{\hbar}\hat{A}}$} &&\Leftrightarrow&&& \left[\hat{H},\hat{A}\right]=0 \\
		\text{Zeitinvarianz} &&\Leftrightarrow&&& \text{Energieerhaltung} \\
		\text{Translationsinvarianz} &&\Leftrightarrow&&& \text{Impulserhaltung} \\
		\text{Rotationsinvarianz} &&\Leftrightarrow&&& \text{Drehimpulserhaltung}
\end{align*}

\section{Translationen}

Definition \& Wirkung, Erzeuger \cite[Glg. 6.12, 6.8, 6.11]{qm}
\begin{equation*}
	\bra{\vec{x}}\hat{T}_{\vec{a}} = \bra{\vec{x}-\vec{a}} \qquad \hat{T}_{\vec{a}}\ket{\vec{x}} = \ket{\vec{x}+\vec{a}} \qquad \hat{T}_{\vec{a}} = e^{-\frac{i}{\hbar}\hat{\vec{P}}\vec{a}}
\end{equation*}
Eigenwerte und -vektoren im Ortsraum \cite[Glg. 6.15]{qm}
\begin{equation*}
	\hat{T}_{\vec{a}}\frac{e^{-\frac{i}{\hbar}\vec{p}\vec{a}}}{(2\pi\hbar)^\frac{3}{2}} = e^{-\frac{i}{\hbar}\vec{p}\vec{a}}\frac{e^{-\frac{i}{\hbar}\vec{p}\vec{a}}}{(2\pi\hbar)^\frac{3}{2}}
\end{equation*}
Nöther (Translationsinvarianz $\Leftrightarrow$ Impulserhaltung) \cite[Glg. 6.16]{qm}
\begin{equation*}
	[\hat{H}, \hat{T}_{\vec{a}}]=0 \qquad \Leftrightarrow \qquad [\hat{H}, \hat{\vec{P}}]=0
\end{equation*}

\section{Parität}

Definition \cite[Glg. 4.32]{qm}
\begin{equation*}
	\hat{S}\ket{\vec{x}} = \ket{-\vec{x}} \qquad \Leftrightarrow \qquad \bra{\vec{x}}\hat{S} = \bra{-\vec{x}}
\end{equation*}
Eigengleichung \cite[Glg. 4.33]{qm}
\begin{equation*}
	\hat{S}\ket{\psi_s} = s\ket{\psi_s} \qquad s=\begin{cases}
		+1 & \text{(gerade)} \\
		-1 & \text{(ungerade)}
	\end{cases}
\end{equation*}

\section{Drehungen}

\begin{itemize}
    \item Alles auch für $\hat{\vec{L}}$ bzw. $\hat{\vec{S}}$ bei spinlosen bzw. punktförmigen Teilchen. \cite[S. 202f]{qm}
\end{itemize}
Erzeuger \cite[Glg. 6.22]{qm}
\begin{equation*}
    \hat{R}_{\vec{n}}(\varphi) = e^{-\frac{i}{\hbar}\varphi \vec{n} \hat{\vec{J}}}
\end{equation*}
Für Spins \cite[Glg. 6.23]{qm}
\begin{equation*}
	e^{-\frac{i}{\hbar}\varphi\vec{n}\hat{\vec{S}}} = \cos\frac{\varphi}{2}-i\vec{n}\vec{\sigma}\sin\frac{\varphi}{2}
\end{equation*}
Bahndrehimpuls- \& (Gesamt-)Drehimpulsoperator (Definition) \cite[Glg. 6.29, 6.30]{qm}
\begin{equation*}
    \hat{\vec{L}} = \hat{\vec{Q}} \times \hat{\vec{P}} \qquad \hat{\vec{J}} = \hat{\vec{L}} + \hat{\vec{S}}
\end{equation*}
Leiteroperatoren (Definition) \cite[Glg. 6.35]{qm}
\begin{equation*}
    \hat{J}_\pm = \hat{J}_x \pm i \hat{J}_y \qquad \hat{J}_x = \frac{\hat{J}_+ + \hat{J}_-}{2} \qquad \hat{J}_y = \frac{\hat{J}_+ - \hat{J}_-}{2i}
\end{equation*}
Wirkung \cite[Glg. 6.52, 6.57, 6.41]{qm}
\begin{align*}
	\hat{\vec{J}}^2\ket{j, m} &= \hbar^2j(j+1)\ket{j,m} & \hat{J}_z\ket{j, m} &= \hbar m\ket{j,m} \\
	\hat{J}_\pm\ket{j, m} &= \hbar\sqrt{j(j+1)-m(m\pm 1)}\ket{j,m\pm 1} & \hat{S}_\pm\ket{\mp z}&=\ket{\pm z} \quad \hat{S}_\pm\ket{\pm z}=0
\end{align*}
Eigenschaften \cite[Glg. 6.31, 6.32, 6.36]{qm}
\begin{gather*}
	\left[\hat{J}_j, \hat{J}_k\right] = i\hbar \epsilon_{jkl} \hat{J}_l \qquad \hat{\vec{J}} \times \hat{\vec{J}} = i\hbar \hat{\vec{J}} \\
	\left[ \hat{J}_z, \hat{J}_\pm \right] = \pm \hbar \hat{J}_\pm \qquad \left[ \hat{J}_+, \hat{J}_- \right] = 2\hbar \hat{J}_z \qquad \left[ \hat{\vec{J}}^2, \hat{J}_\pm \right] = 0
\end{gather*}
Drehimpulsquantenzahl \& magnetische Quantenzahl des Drehimpulsop. \cite[Tab. 6.1]{qm}
\begin{equation*}
    \ket{j,m} \qquad j\in\left\{\frac{n}{2}\mid n\in\mathbb{N}_0\right\} \qquad m\in\left\{-j,\dots,+j\right\}
\end{equation*}
``Skalare'' Operatoren und ``Vektor''-Operatoren \cite[Glg. 6.33, 6.34]{qm}
\begin{equation*}
	[\hat{\vec{J}},\hat{A}]=0 \qquad [\hat{J}_i,\hat{A}_j]=i\hbar\epsilon_{ijk}\hat{A}_k
\end{equation*}


\chapter{Wellenfunktion}

Randbedingungen \cite[Kap. 4.1.2]{qm}
\begin{itemize}
    \item $\psi(x)$ immer stetig
    \item unendliches Potential $\Rightarrow$ $\psi(\vec{x})=0$
    \item endliches Potential $\Rightarrow$ $\psi'(x)$ stetig
    \item unendliche Potentialkante $\Rightarrow$ $\psi'(x)$ unstetig
\end{itemize}
Aufenthaltswahrscheinlichkeitsdichte \cite[Glg. A.97]{qm}
\begin{equation*}
	\rho(\vec{x}, t) = |\psi(\vec{x}, t)|^2
\end{equation*}
Parität \cite[S. 105]{qm}
\begin{itemize}
    \item Bei einem symmetrischen Potential können die Eigenfunktionen von $\hat{H}$ gerade \& ungerade gewählt werden.
\end{itemize}
Reelle Wellenfunktionen \cite[S. 111]{qm}
\begin{equation*}
	\psi(\vec{x},t)\in\mathbb{R} \qquad \Rightarrow \qquad \rho(\vec{x},t)=\rho(\vec{x}) \qquad \Rightarrow \qquad \vec{j}(\vec{x},t)=\vec{0}
\end{equation*}

\section{Eindimensionale Potentialprobleme}

Entartung \cite[Kap. 4.6.3]{qm}
\begin{itemize}
    \item Gebundene Zustände im Eindimensionalen sind nie entartet.
\end{itemize}
Knotensatz \cite[Kap. 4.6.4]{qm}
\begin{itemize}
    \item Die Wellenfunktion von $E_n$ hat $n$ Nullstellen im Eindimensionalen. ($\cdots<E_n<E_{n+1}<\cdots, \ n\in\mathbb{N}_0$)
\end{itemize}
Realität \cite[Kap. 4.6.5]{qm}
\begin{itemize}
    \item Die Wellenfunktionen der gebundenen Eigenzustände eines eindimensionalen Potentialproblems ohne Magnetfeld können immer reell gewählt werden.
\end{itemize}

\subsection{Spezielle Potentiale}

Konstantes Potential $V(x)=V$ \cite[Glg. 4.5]{qm}
\begin{gather*}
	\psi(x)=ae^{+ikx}+be^{-ikx}=ae^{+\kappa x}+be^{-\kappa x} \\
	k=\sqrt{\frac{2m}{\hbar^2}(E-V)} \qquad \kappa=\sqrt{\frac{2m}{\hbar^2}(V-E)}=ik
\end{gather*}
Unendlicher Potentialtopf $V(x)=\begin{cases} V_0 & x\in[0,L] \\ \infty & x\not\in[0,L] \end{cases}$ \cite[Glg. 4.7-4.9]{qm}
\begin{equation*}
	\psi_n(x)=\begin{cases}
		\sqrt{\frac{2}{L}}\sin k_nx & x\in[0,L] \\
		0                           & x\not\in[0,L]
	\end{cases} \qquad k_n=\frac{n\pi}{L} \qquad E_n=\frac{\hbar^2\pi^2}{2mL^2}n^2+V_0 \qquad n\in\mathbb{N}^+
\end{equation*}
Potentialbarriere $V(x)=\begin{cases} V_0 & x\in[-L/2,+L/2] \\ 0 & x\not\in[-L/2,+L/2] \end{cases}$ \cite[Glg. 4.53, 4.48, 4.49, 4.54]{qm}
\begin{gather*}
	\psi(\vec{x},t) = \begin{cases}
		e^{+ikx}+Ae^{-ikx} & x\leq-L/2 \\
		B_1e^{+\kappa x} + Be^{-\kappa x} & -L/2\leq x\leq+L/2 \\
		Ce^{+ikx} & +L/2\leq x
	\end{cases} \\
	\begin{aligned}
		k &= \sqrt{\frac{2mE}{\hbar^2}} & \kappa &= \sqrt{\frac{2m}{\hbar^2}(V_0-E)} \\
		A &= \frac{1}{Z}e^{-ikL}(1+\rho^2)\sinh\kappa L & B_1 &= -\frac{1}{Z}e^{-ikL/2}(1-i\rho)e^{-\kappa L/2} \\
		C &= \frac{1}{Z}i2\rho e^{-ikL} & B_2 &= +\frac{1}{Z}e^{-ikL/2}(1+i\rho)e^{+\kappa L/2} \\
			\rho &= \frac{\kappa}{k} & Z &= (1-\rho^2)\sinh\kappa L + 2i\rho\cosh\kappa L \\
		T &= |C|^2 & R &= |A|^2
	\end{aligned}
\end{gather*}

\section{Basen}

Hermite-Funktionen \& Hermite-Polynome \begin{refsection}\footfullcite[Bsp. 2.16ii]{funki}\end{refsection}
\begin{equation*}
	h_n(x)=\frac{e^{-\frac{x^2}{2}}}{\sqrt{2^nn!\sqrt{\pi}}}H_n(x) \qquad H_n(x)=(-1)^ne^{x^2}\frac{d^n}{dx^n}e^{-t^2}
\end{equation*}
Kugelflächenfunktionen \& zugeordnete Legendre-Polynome \cite[Glg. 6.59]{qm}
\begin{gather*}
	\braket{\theta,\varphi|l,m} = Y_l^m(\theta,\varphi) = (-1)^m\sqrt{\frac{(2l+1)(l-m)!}{4\pi(l+m)!}}e^{im\varphi}P_l^m(\cos\theta) \\
	P_l^m(\cos\theta) = \frac{\sin^m\theta}{2^ll!}\frac{d^{l+m}}{d(\cos\theta)^{l+m}}\sin^{2l}\theta
\end{gather*}


\chapter{Störungstheorie}

\input{chapters/7_störungstheorie}



\end{document}
